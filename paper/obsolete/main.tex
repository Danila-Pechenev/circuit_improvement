\documentclass[12pt,letterpaper]{article}
\usepackage[utf8]{inputenc}

\usepackage{fullpage}
\usepackage{paratype}
\usepackage{eulervm}
\usepackage{microtype}

\usepackage{amsmath}
\usepackage{amsthm}
\usepackage{amsfonts}
\usepackage{amssymb}
\newtheorem{theorem}{Theorem}
\newtheorem{lemma}{Lemma}

\bibliographystyle{plainurl}

\input defs

\title{SAT-based Circuit Local Improvement}
\author{Alexander~S. Kulikov \and Nikita Slezkin}

\begin{document}
\input body
\end{document}



\section{Overview}
Let $l=\lceil \log_2(n+1)\rceil$. The function $\SUM_n \colon \{0,1\}^n \to \{0,1\}^{l}$ computes the binary representation of~the sum of~$n$~bits:
$\SUM_n(x_1, \dotsc, x_n)=(w_0, w_1, \dotsc, w_{l-1})$ such that
\[\sum_{i=1}^{n}x_i=\sum_{i=0}^{l-1}2^iw_i \, .\]
This way, the function transform $n$~bits of weight~0 into $l$~bits 
of~weights $(0,1,\dotsc,l-1)$.

This is~a~fundamental symmetric function. In~particular, for any symmetric predicate~$f \colon \{0,1\}^n \to \{0,1\}$,
\[\size(f) \le \size(\SUM_n)+o(n) \, ,\]
where $\size(\cdot)$ is~the circuit size (in~this text, our default computational model is Boolean circuits over the full binary basis).



\section{Circuit Size of $\SUM_n$}%{Circuit Size for Small Input Sizes}
\subsection{Three Inputs}
The exact circuit size of $\SUM_n$ is known for all~$n$ up to~$n=5$. The table below shows upper bounds for some other values of~$n$:
\begin{center}
\begin{tabular}{lccccccccccc}
\toprule
$n$ & $2$ & $3$ & $4$ & $5$ & $6$ & $7$ & $8$ & $9$ & $10$ & $15$ & $31$\\
\midrule
$\size(\SUM_n)$ & $2$ & $5$ & $9$ & $11$ & $16$ & $19$ & $25$ & $ 27$ & $33$ & $53$ & $125$\\
\bottomrule
\end{tabular}
\end{center}
%
Below, we show optimal circuits for $n=2,3$. They are known as half adder and full adder, respectively.
%
\begin{mypic}
\begin{scope}[yscale=.8]
%\draw[help lines] (0,0) grid (16,6);

\begin{scope}[yshift=2cm]
\foreach \n/\x/\y in {1/0/1, 2/1/1}
  \node[input] (x\n) at (\x, \y) {$x_{\n}$};
\node[gate, label=left:$w_1$] (g1) at (0,0) {$\land$};
\node[gate, label=right:$w_0$] (g2) at (1,0) {$\oplus$};
\foreach \f/\t in {x1/g1, x1/g2, x2/g1, x2/g2}
  \draw[->] (\f) -- (\t);
\end{scope}

\begin{scope}[xshift=4cm]
\foreach \n/\x/\y in {1/0/3, 2/1/3, 3/2/3}
  \node[input] (x\n) at (\x, \y) {$x_{\n}$};
\node[gate] (g1) at (0.5,2) {$\oplus$};
\node[gate] (g2) at (1.5,2) {$\oplus$};
\node[gate] (g3) at (0.5,1) {$\lor$};
\node[gate, label=right:$w_0$] (g4) at (1.5,1) {$\oplus$};
\node[gate, label=right:$w_1$] (g5) at (0.5,0) {$\oplus$};
\foreach \f/\t in {x1/g1, x2/g1, x2/g2, x3/g2, g1/g3, g2/g3, g1/g4, g3/g5, g4/g5}
  \draw[->] (\f) -- (\t);
\path (x3) edge[bend left,->] (g4);
\end{scope}
\end{scope}
\end{mypic}

\subsection{From Three Inputs to~the General Case}
These two basic blocks can already be~used to~compute $\SUM_n$ for any~$n$.
E.g., one can compute $\SUM_5$ by a~circuit of~size~12 as~follows:
%
\begin{mypic}
\begin{scope}[yscale=.7]
%\draw[help lines] (0,-5) grid (16,6);
\foreach \n in {1,...,5}
  \node[input] (\n) at (\n,6) {$x_{\n}$};
\draw (0.5, 5.5) rectangle (3.5, 4.5); \node at (2, 5) {$\SUM_3$};
\foreach \n in {1, 2, 3}
  \draw[->] (\n) -- (\n, 5.5);
\draw (2.5, 3.5) rectangle (5.5, 2.5); \node at (4, 3) {$\SUM_3$};
\path (3, 4.5) edge[->] node[l] {0} (3, 3.5);
\foreach \n in {4, 5}
  \draw[->] (\n) -- (\n, 3.5);
\draw (1.5, 1.5) rectangle (3.5, 0.5); \node at (2.5, 1) {$\SUM_2$};
\path (2, 4.5) edge[->] node[l] {1} (2, 1.5);
\path (3, 2.5) edge[->] node[l] {1} (3, 1.5);
\node[input] (w2) at (2,-1) {$w_2$};
\node[input] (w1) at (3,-1) {$w_1$};
\node[input] (w0) at (4.5,-1) {$w_0$};
\path (2, 0.5) edge[->] node[l] {1} (w2);
\path (3, 0.5) edge[->] node[l] {0} (w1);
\path (4.5, 2.5) edge[->] node[l] {0} (w0);
\end{scope}
\end{mypic}

The same trick shows that $\size(\SUM_n) \le 5n+o(n)$. Indeed, 
the $\SUM_3$ circuit replaces three bits of weight~$i$ by two bits of weight~$i$ and $i+1$. Thus, to~get from $n$~bits of weight~$0$ to $l$~bits of weight $(0,\dotsc, l-1)$, one may 
repeatedly apply $\SUM_3$ to three bits of the same weight. 
An~application of~$\SUM_3$ reduces the number of~bits by~one, hence
the number of such applications is~at~most~$n$. What will remain after this step is at most two bits on each of $l=o(n)$~levels. Pictorially, this looks as~follows.

\begin{mypic}
    %\mydefgrid
    \foreach \n/\t/\x in {{x1/x_1/0.5}, {x2/x_2/1}, {x3/x_3/2}, {x4/x_4/2.5}, {x5/x_5/3.5}, {x6/x_6/4}, {x7/x_7/5}, {x8/x_8/5.5}, {x10/x_n/8.5}}
    {
      \node[] (\n) at (\x,6) {$\t$};
      \draw[->] (\n) -- (\x,5.25);
    }
    \foreach \n/\t/\y in {{y1/w_0/4.75}, {y2/w_1/3}, {y3/w_{l-1}/0.5}}
      \node[] (\n) at (9.5,\y) {$\t$};
    \foreach \l/\r/\x/\y in {{0.25/5.25/1.25/4.25}, {1.75/5.25/2.75/4.25}, {3.25/5.25/4.25/4.25}, {4.75/5.25/5.75/4.25}, {7.75/5.25/8.75/4.25}, {1/3.5/2/2.5}, {4/2.5/5/3.5}, {4/0/5/1}}
      \draw (\l,\r) rectangle (\x,\y);
    \foreach \x/\y in {{0.75/4.75}, {2.25/4.75}, {3.75/4.75}, {5.25/4.75}, {8.25/4.75}, {1.5/3}, {4.5/3}, {4.5/0.5}}
      \node[] at (\x,\y) {\small $\SUM_3$};
    \foreach \x/\y/\l/\r in {{1.25/4.75/1.75/4.75}, {2.75/4.75/3.25/4.75}, {4.25/4.75/4.75/4.75}, {5.75/4.75/6.25/4.75}, {2/3/4/3}, {5/3/7/3}, {0.75/4.25/1.25/3.5}, {2.25/4.25/1.75/3.5}, {3.75/4.25/4.25/3.5}, {5.25/4.25/4.75/3.5}, {5.25/1.75/4.75/1.0}, {3.75/1.75/4.25/1.0}, {7.25/4.75/7.75/4.75}}
      \draw[->] (\x,\y) -- (\l,\r);
    \foreach \x/\y/\t in {{6.75/4.75/\cdots}, {7.74/3/\cdots}, {4.5/2/\cdots}}
      \node[] at (\x,\y) {$\t$};
    \foreach \x/\y/\v in {{8.75/4.75/y1}, {8.25/3/y2}, {5/0.5/y3}}
      \draw[->] (\x,\y) -- (\v);
\end{mypic}

In~general, this gives the following upper bound:
if $\size(\SUM_k) \le r$, then for all~$n$, 
\begin{equation}\label{eq:master}
\size(\SUM_n) \le \frac{rn}{k-\lceil \log_2(k+1) \rceil}+o(n) \, .
\end{equation}

\subsection{Five Inputs and Improved General Case}
It~turns out that 
there exists a~smaller circuit for $\SUM_5$
and that it can be used in two different ways
to~improve the general upper bound! The following circuit of size~11 was found by Knuth. This is how he describes it: ``But $s(5)=12$ is \emph{not} optimum, despite the beauty of~7.1.2-(29)!'' (Here, 7.1.2-(29) refers to~the circuit of size~12.)

\begin{mypic}
\begin{scope}[yscale=.8, label distance=-1mm]
\foreach \n/\x/\y in {1/0/3, 2/1/3, 3/2/3, 4/2.5/1, 5/3.5/1}
  \node[input] (x\n) at (\x, \y) {$x_{\n}$};
\node[gate,label=left:$g_1$] (g1) at (0.5,2) {$\oplus$};
\node[gate,label=left:$g_2$] (g2) at (1.5,2) {$\oplus$};
\node[gate,label=left:$g_3$] (g3) at (0.5,1) {$\lor$};
\node[gate,label=left:$g_4$] (g4) at (1.5,1) {$\oplus$};
\node[gate,label=left:$g_5$] (g5) at (0.5,0) {$\oplus$};
\node[gate,label=left:$g_6$] (g6) at (2,0) {$\oplus$};
\node[gate,label=left:$g_7$] (g7) at (3,0) {$\oplus$};
\node[gate,label=left:$g_8$] (g8) at (2,-1) {$>$};
\node[gate, label=right:$w_0$] (g9) at (3,-1) {$\oplus$};
\node[gate, label=right:$w_1$] (g10) at (2,-2) {$\oplus$};
\node[gate, label=right:$w_2$] (g11) at (1.5,-3) {$>$};

\foreach \f/\t in {x1/g1, x2/g1, x2/g2, x3/g2, g1/g3, g2/g3, g1/g4, g3/g5, g4/g5, x4/g6, g4/g6, x4/g7, x5/g7, g6/g8, g7/g8, g7/g9, g3/g10, g8/g10, g10/g11, g5/g11}
  \draw[->] (\f) -- (\t);

\path (x3) edge[->,bend left] (g4);
\path (g4) edge[->,bend left=20] (g9);

\end{scope}
\end{mypic}
%
Using this circuit, one can build a~circuit of size $5+11+2+1=19$ for $\SUM_7$:
\begin{mypic}
\begin{scope}[yscale=.7]
%\draw[help lines] (0,-5) grid (16,6);
\foreach \n in {1,...,7}
  \node[input] (\n) at (\n,6) {$x_{\n}$};
\draw (0.5, 5.5) rectangle (3.5, 4.5); \node at (2, 5) {$\SUM_3$};
\foreach \n in {1, 2, 3}
  \draw[->] (\n) -- (\n, 5.5);
\draw (2.5, 3.5) rectangle (7.5, 2.5); \node at (5, 3) {$\SUM_5$};
\path (3, 4.5) edge[->] node[l] {0} (3, 3.5);
\foreach \n in {4, 5, 6, 7}
  \draw[->] (\n) -- (\n, 3.5);
\draw (1.5, 1.5) rectangle (3.5, 0.5); \node at (2.5, 1) {$\SUM_2$};
\path (2, 4.5) edge[->] node[l] {1} (2, 1.5);
\path (3, 2.5) edge[->] node[l] {1} (3, 1.5);
\draw (2.5, -0.5) rectangle (4.5, -1.5); \node at (3.5, -1) {XOR};
\path (4, 2.5) edge[->] node[l] {2} (4, -0.5);
\path (3, 0.5) edge[->] node[l] {1} (3, -0.5);

\node[input] (w0) at (6,-3) {$w_0$};
\node[input] (w2) at (3.5,-3) {$w_2$};
\node[input] (w1) at (2,-3) {$w_1$};

\path (6, 2.5) edge[->] node[l] {0} (w0);
\path (2, 0.5) edge[->] node[l] {0} (w1);
\path (3.5, -1.5) edge[->]  (w2);
\end{scope}
\end{mypic}
%
Plugging $\size(\SUM_7) \le 19$ into~\eqref{eq:master}, gives an~upper bound $\size(\SUM_n) \le 4.75n$.

Interestingly, the same $\SUM_5$ circuit can be used to~get a~$4.5n$ upper bound! 

For this, 
%


Using this block, one can compute $\SUM_n$ by a~circuit of size $4.5n$ as~follows:

\begin{mypic}
    \foreach \n/\t/\x in {{x1/x_1/0.5}, {x2/x_2/1}, {x3/x_3/1.5}, {x4/x_4/2}, {x5/x_5/3}, {x6/x_6/3.5}, {x7/x_7/4}, {x8/x_8/4.5}, {x9/x_{n-1}/7.75}, {x10/x_n/8.5}}
      \node[] (\n) at (\x,6.5) {$\t$};
    \foreach \n/\x/\a/\b in {{xor1/1/x1/x2}, {xor2/2/x3/x4}, {xor3/3.5/x5/x6}, {xor4/4.5/x7/x8}, {xor5/8.5/x9/x10}}
    {
      %\mysimplegatetwo{\n}{(\x,5.75)}{$\oplus$}{\a}{\b};  
      \node[gate] (\n) at (\x, 5.75) {$\oplus$};
      \draw[->] (\a) -- (\n);
      \draw[->] (\b) -- (\n);
      \path[draw,->] (\n) -- (\x,5.25);
    }
    \foreach \n/\x in {x1/0.5, x3/1.5, x5/3, x7/4, x9/7.75}
      \path[draw,->] (\n) -- (\x,5.25);
    \foreach \n/\t/\y in {{y1/w_0/4.75}, {y2/w_1/3}, {y3/w_{l-1}/0.5}}
      \node[] (\n) at (9.5,\y) {$\t$};
    \foreach \l/\r/\x/\y in {{0.25/5.25/2.25/4.25}, {0.25+2.5/5.25/2.25+2.5/4.25}, {0.25+6.5/5.25/2.25+6.5/4.25}, {1.5/2.5/3.5/3.5}, {3.5/0/5.5/1}}%, {1.75/5.25/2.75/4.25}, {3.25/5.25/4.25/4.25}, {4.75/5.25/5.75/4.25}, {7.75/5.25/8.75/4.25}, {1/3.5/2/2.5}, {4/2.5/5/3.5}, {4/0/5/1}}
      \draw (\l,\r) rectangle (\x,\y);
    \foreach \x/\y in {{1.25/4.75}, {3.75/4.75}, {7.75/4.75}, {2.5/3}, {4.5/0.5}}
      \node[] at (\x,\y) {MDFA};
    \foreach \xa/\xb in {0.75/1.75, 1.25/2.25, 3.75/2.75, 4.25/3.25}
      \draw[->] (\xa,4.25) -- (\xb,3.5);
    \foreach \x/\y/\t in {{5.75/4.75/\cdots}, {6.75/3/\cdots}, {4.5/1.75/\cdots}}
      \node[] at (\x,\y) {$\t$};
    \foreach \x/\y/\v in {{8.75/4.75/y1}, {8.25/3/y2}, {5.5/0.5/y3}}
      \draw[->] (\x,\y) -- (\v);
    \foreach \xa/\xb/\y in {2.25/2.75/4.75, 4.75/5.25/4.75, 6.25/6.75/4.75, 3.5/4.5/3}
      \draw[->] (\xa,\y) -- (\xb,\y);
\end{mypic}

Finally, the block MDFA is contained in the circuit of size~11 for $\SUM_5$! 










\end{document}